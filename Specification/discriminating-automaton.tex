Terminology is preliminary.

The AMOP says that when the second value (called OK) returned by a
call to COMPUTE-APPLICABLE-METHODS-USING-CLASSES is false, the nature
of the first value returned is not specified.  For the technique
described here to work, in such a situation, we need for the first
return value to be a list of all the methods that might be applicable,
and we need for them to be ordered from most specific to least
specific.  We will say that the call to
COMPUTE-APPLICABLE-METHODS-USING-CLASSES FAILS, when we mean that the
second return value is false.  Otherwise, we say that that it
succeeds.

Suppose we have a generic function F with a method combination MC.  It
is possible to add a method M to F such thatM has qualifiers that are
incompatible with MC.  Such incompatible qualifiers are detected when
the method combination is invoked, i.e., when COMPUTE-EFFECTIVE-METHOD
is called, simply because the method combination is the place where
qualifiers are analyzed.  The method combination is generally opaque,
So in general it is not possible to detect incompatible qualifiers
when a method is added to a generic function.  But for pre-defined
method combinations, it is obviously possible for an implementation to
detect incompatible qualifiers when a method is added to a generic
function that currently has such a method combination.  However, an
attempt to add a method with incompatible qualifiers for the current
method combination of the generic function can not be rejected with an
error.  The reason is that the generic function can be reinitialized
after the method has been added, and such reinitialization can result
in a new method combination for which the previously incompatible
qualifier is now compatible.

The AMOP says that if COMPUTE-APPLICABLE-METHODS-USING-CLASSES fails,
we must call COMPUTE-APPLICABLE-METHODS instead, in order to determine
a list of applicable methods for this particular combination of
argument values.  The AMOP furthermore tells us that we are not
allowed to cache the result of this call.  As a result, when EQL
specializers are involved, it is often not possible to cache the
effective methods, resulting in very slow generic-function
invocations.  However, there is a way out.  The AMOP also says (and
this is a general rule in object-oriented programming), client-defined
methods on specified generic functions must not be applicable when
given only arguments with specified classes.  As a result, it is not
possible for conforming client code to detect whether
COMPUTE-APPLICABLE-METHODS is called with a direct instance of
STANDARD-GENERIC-FUNCTION.  We take advantage of this possibility to
cache effective method functions even when EQL specializers are
involved.

Let's suppose we call a generic function with a list of required
arguments such as the call to COMPUTE-APPLICABLE-METHODS-USING-CLASSES
fails.  And suppose that the subsequent call to
COMPUTE-APPLICABLE-METHODS succeeds.  Then we would like to cache the
resulting effective-method function.  The reason the call to
COMPUTE-APPLICABLE-METHODS-USING-CLASSES failed, is that there are
methods involved that have EQL specializers, and that information
about the identity of the arguments, in addition to their classes, is
required in order to determine the applicable methods.  If we want to
cache effective methods in situations like this, one way to do it is
to enter the exact argument identities (as EQL specializers) in the
call history, and it would never be an error to do so.  However, we
would then often fill up the call history with a huge number of
combinations of required arguments.  So we must find another way.  But
there are complications.

Suppose again that the call to
COMPUTE-APPLICABLE-METHODS-USING-CLASSES failed, and that the call to
COMPUTE-APPLICABLE-METHODS returned a list of methods with no EQL
specializers.  We can not cache this result by using only the classes
of the arguments.  Since the call to
COMPUTE-APPLICABLE-METHODS-USING-CLASSES failed, there are methods
with EQL specializers that are applicable to SOME argument values of
the these classes.  If we do not take these methods into account, an
incorrect effective method will be invoked when a subsequent call
involves argument values corresponding to some such EQL specializers.
Even when the result of calling COMPUTE-APPLICABLE-METHODS contains
one or more methods with EQL specializers, but not for every argument
position, we can not cache the result using only the class of the
remaining argument positions.  As a result, the technique described
here, takes into account all possible effective methods when a call to
COMPUTE-APPLICABLE-METHODS-USING-CLASSES fails,

The CALL HISTORY is a list of call-history ENTRIES.  An entry has a
KEY and some additional information.  The key is a list with as many
elements as there are true values in the specializer profile.  An
element of the key can be either a class or an EQL specializer.  The
key is used when a method is added or removed, to determine whether
the entry is still valid, or should be removed.  It is also used to
compute the discriminating function.  When the call history contains
an entry E1 for which there is a class C in some position, and there is
another entry E2 that contains an EQL specializer with an object O such
that the class of O is C (i.e., O is a direct instance of C), then the
meaning of these entries is that, if they are both valid for a
particular list of required arguments, then E2 applies, but not E1.
However, if E1 and E2 are not both valid for some list of required
arguments,  because they are incompatible in some other argument
position, then E1 is valid when the argument is O.  For example, if we
have two entries E1 = (fixnum character) and E2 = (3 character), then
E1 is valid only when the first argument is not 3.  and if we have 
E1 = (fixnum character) and E2 = (3 number), then E1 is valid even
when the first argument is equal to 3.  In other words, if we test the
first argument and determine that it is equal to 3, we can not exclude
E1 until we have tested the second argument as well.

The additional information in an entry consists of a list of
applicable methods and an effective method function computed from that
list.  The list of applicable methods may have been modified after it
was computed by COMPUTE-APPLICABLE-METHODS-USING-CLASSES, in that an
accessor method may have been replaced in order to account for the
location of the slot being accessed, as a function of the information
in the entry key.  This information is bundled up onto a CONS cell,
where the list of applicable methods is in the CAR and the effective
method function is in the CDR. Several entries may share this
information.  When that is the case, their CONS cells are identical
(EQ).

Entries in the call history may be removed as a result of changes in
the class hierarchy, or as a result of a method being added or removed
from the generic function.  How the call history is updated in these
cases is described elsewhere.

Entries in the call history are added as a result of a dispatch miss
that involves arguments that can invoke some effective method, but
this particular combination of arguments has not yet been encountered.
When a dispatch-miss happens, we call
COMPUTE-APPLICABLE-METHODS-USING-CLASSES, passing it the list of
classes of the required arguments.  This call may either succeed or
fail.

If it succeeds, we take the list of methods returned in the
first return value, and we call COMPUTE-EFFECTIVE-METHOD, using the
method combination of the generic function.  If that call fails, we
signal an error.  Otherwise, we create a new call-history entry with
the list of argument classes as the key.  We examine the list of
applicable methods to determine whether there is an accessor method in
that list.  If that is the case, then that accessor method is replaced
by one that accesses the slot directly.  The location of the slot is
determined by the corresponding class of the relevant argument.  The
possibly modified list of applicable methods is then compared to
existing such lists in other entries of the call history.  If an entry
is found with the same applicable methods as the new entry, then the
CONS cell is shared with the existing entry.  If not, an
effective-method function is computed and entered into a new CONS cell
together with the list of the applicable methods.

If the call to COMPUTE-APPLICABLE-METHODS-USING-CLASSES fails, we
first call COMPUTE-APPLICABLE-METHODS with the arguments to the
generic function.  We then call COMPUTE-EFFECTIVE-METHOD with the
resulting list of applicable methods.  If the call to
COMPUTE-EFFECTIVE-METHOD fails, we signal an error. 

If COMPUTE-EFFECTIVE-METHOD succeeds, we need to cache this result.
Suppose that for each required argument position in the specializer
profile, at least one method in the list returned by
COMPUTE-APPLICABLE-METHODS has an EQL specializer.  Then this list of
methods is valid only for the particular combination of required
arguments in this call.  In this case, we can add an entry to the call
history containing an EQL specializer in every position in the key.
As mentioned above, it is always safe to add such an entry, but in
this case, such an entry will have to be present, because this
particular list of methods is applicable only for this particular
combination of required arguments.

Now suppose again that COMPUTE-EFFECTIVE-METHOD succeeds, but there is
at least one argument position that does not contain an EQL
specialzier in any of the methods returned by
COMPUTE-APPLICABLE-METHODS. Then things get more complicated.
Consider for example a call with two required arguments 1 and 2.  And
suppose COMPUTE-APPLICABLE-METHODS returned two methods with
specializers (1 fixnum) and (fixnum fixnum) respectively.  We must
find a set of keys to enter into the call history that works
correctly.  We can not enter only an entry with the key (1 fixnum),
because it is possible that there is a method with specializers
(fixnum 3) that was not included in the list returned by 
COMPUTE-APPLICABLE-METHODS.  Clearly, the new entries in the call
history must reflect the presence of this method in case some
subsequent call involves arguments 1 and 3.

So we need to examine the list of methods in the first return value of
the call to COMPUTE-APPLICABLE-METHODS-USING-CLASSES in order to
determine combinations of specializers to enter into the call history.

Specifically, for each method, we compute a SPECIALIZER-SIGNATURE.
This signature is a list with the same length as the number of true
elements in the specializer profile.  It contains the corresponding
EQL specializer in each position that has an EQL specializer, and the
symbol * in other positions. We collect the list of all such
signatures, and make sure it has no duplicates.  For each unique
signature in the list, we filter the methods in the list of applicable
methods, and keep only the ones that are applicable to that particular
signature.  When * is replaced by the class of the corresponding
argument in that position, we have a key for an entry in the call
history.  But we are not done yet.  For each filtered list of methods,
we must call COMPUTE-EFFECTIVE-METHOD, to determine whether it
succeeds.  If it does succeed, we associate the new entry with the
resulting effective method.  If it fails, however, we can not simply
exclude this particular entry.  Suppose again that we have a call with
arguments 1 and 2, and methods with specializers (1 fixnum) [a],
(fixnum 3) [b] (fixnum fixnum) [c], and (fixnum integer) [d].  We
have specializer signatures (1 *) (* 3) and (* *) generating entries
(1 fixnum) [a, c, d], (fixnum 3) [b, c, d], and (fixnum fixnum) [a, b,
c, d].  suppose further that the call to COMPUTE-EFFECTIVE-METHOD
fails when given methods b, c, and d.  If we exclude the entry with
the key (fixnum 3), we will incorrectly recognize a call with
arguments 2 and 3, as an instance of the key (fixnum fixnum).  We must
therefore include this entry with an effective method that signals an
error.

The construction of the automaton described below is not quite
finished.

The call history is used to construct an AUTOMATON.  The final states
of the automaton are the effective method functions present in the
call history, with no duplication.  The automaton is built by a
recursive function BUILD-AUTOMATON.  Each invocation of
BUILD-AUTOMATON corresponds to a state of the automaton.  The function
has a bunch of parameters.

One parameter is a list called the ARGUMENT-STATE, each element
corresponding to an argument to the generic function that must be
tested.  An element of the argument state can be either NIL (the start
value), :BUILT-IN, or :STANDARD.  NIL indicates that in the current
state of the automaton, the corresponding argument may contain any
kind of object.  :BUILT-IN and :STANDARD indicate that the
corresponding argument has been tested to determine whether it
contains a an instance of a BUILT-IN class or an instance of
STANDARD-OBJECT, and the value determines what state the automaton is
in, in the current invocation of BUILD-AUTOMATON.

Another parameter is a list of keys of entries in the call history.
The keys are copied (COPY-LIST) before they are given to
BUILD-AUTOMATON.  An invocation of BUILD-AUTOMATON chooses an argument
position to examine in the corresponding state of the automaton.  The
choice is not important for correctness, but might be for performance.
In each state, a DISCRIMINATING STEP is chosen, and this step
determines which of two successor states becomes the new current
state.  A discriminating step can be either an OBJECT STEP or a CLASS
STEP.  An object step can be chosen only if the selected argument
position has an object in at least one of the keys.  A class step can
always be chosen, but might not always be a good choice.

An object step consists of generating two transitions based on a
comparison (EQL) of the argument to a literal object.  The TRUE branch
will enter a state that contains the keys with that object and with
the class of that object.  The false branch will contain every key
except the ones with the object tested for.

The purpose of a class step is to segregate the keys based on the
class of the corresponding argument.  If the ARGUMENT-STATE contains
NIL for the chosen position, only one type of step is permitted,
namely one based on the tag for standard objects.  The true branch
will lead to a state where every class is a built-in class, and the
false state will contain the remaining keys. If a key contains an
EQL object, the class of that object determines which state it goes
to.  The recursive invocation of BUILD-AUTOMATON will contain
:BUILT-IN for one state and :STANDARD for the other, 

If the ARGUMENT-STATE contains :BUILT-IN in the chosen position, then
the tag bits for some class in one of the keys is tested for. [to be
continued] 

If BUILD-AUTOMATON chooses to select on classes, and the argument
state is :STANDARD, it uses the sorted list of unique numbers of each
class so that it splits the list in as close to two halves based on
the stamp of the object. [to be continued]

The resulting automaton is minimized so that equivalent states are
merged.

From the automaton, we generate a DISCRIMINATING TAGBODY.

Satiation.  The description of satiation is not finished.

Recall that the goal of satiation is to make a generic function
sufficiently operational that metastability problems are avoided.
Therefore, satiation does not have to be total, in that the call
history does not have to contain every possible argument combination
that results in an effective method.  We examine every method, and we
access the list of its specializers, restricted to the arguments that
are present in the specializer-profile.  For each specialized
parameter we create a list of specializers from the specializers of
the methods.  Any specializer that is the class named T is excluded.
And any EQL specializer is replaced by the class of its object.
Duplicates in each list are removed, and each list is augmented by
every defined subclass of every member class.  Then a list of types is
created.  Each tuple is a list of the same length as the specializer
profile.  The tuples in the list cover every possible combination of
classes from each class list.  Each tuple is then used in the process
above, as if it were the classes of some arguments given, thereby
filling up the call history. Finally, the discriminating function is
constructed as above.

Example, for TYPEP-ATOMIC:

Method specializers:
1.  (t symbol)
2.  (t class)
3.  (t (EQL atom))
4.  (cons (EQL atom))
5.  (t (EQL base-char))
6.  (character (EQL base-char))
7.  (t (EQL standard-char))
8.  (character (EQL standard-char))
9.  (t (EQL keyword))
10. (symbol (EQL keyword))
11. (t (EQL nil))

initial first list:
(cons character symbol)
initial second list:
(symbol class)

None of the classes have any subclasses.

tuples:
(cons symbol)
(cons class)
(character symbol)
(character class)
(symbol symbol)
(symbol class)

for (cons symbol)
   compute-applicable-methods-using-classes returns:
   4.  (cons (EQL atom))
   3.  (t (EQL atom))
   5.  (t (EQL base-char))
   7.  (t (EQL standard-char))
   9.  (t (EQL keyword))
   11. (t (EQL nil))
   1.  (t symbol)
   OK = false

   Specializer signatures:
   a. (* atom)
   b. (* base-char)
   c. (* standard-char)
   d. (* keyword)
   e. (* nil)
   f. (* *)

   Filter using (* atom):
   4.  (cons (EQL atom))
   3.  (t (EQL atom))
   1.  (t symbol)

   Create entry: with key (cons (EQL atom))
   4.  (cons (EQL atom))
   3.  (t (EQL atom))
   1.  (t symbol)

   Filter using (* base-char):
   5.  (t (EQL base-char))
   1.  (t symbol)

   Create entry: with key (cons (EQL base-char))
   5.  (t (EQL base-char))
   1.  (t symbol)

   Filter using (* standard-char):
   7.  (t (EQL standard-char))
   1.  (t symbol)

   Create entry: with key (cons (EQL standard-char))
   7.  (t (EQL standard-char))
   1.  (t symbol)

   Filter using (* keyword):
   9.  (t (EQL keyword))
   1.  (t symbol)

   Create entry: with key (cons (EQL keyword))
   9.  (t (EQL keyword))
   1.  (t symbol)

   Filter using (* nil):
   11. (t (EQL nil))
   1.  (t symbol)

   Create entry: with key (cons (EQL nil))
   11. (t (EQL nil))
   1.  (t symbol)

   Filter using (* *):
