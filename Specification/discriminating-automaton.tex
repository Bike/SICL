Names are preliminary.

The CALL HISTORY is a list of CALL-HISTORY ENTRIES.  An entry has a
KEY and some additional information.  The key is a list with as many
elements as there are true values in the specializer profile.  An
element of the key can be either a class or an EQL specializer object.
To distinguish between the two, we can for instance represent them as
a CONS: (class . <class>) (object . <object>).  The key is used when a
method is added or removed, to determine whether the entry is still
valid, or should be removed.  But the key is not what is used to
compute the discriminating function.

The additional information includes a list of applicable methods and
an effective method function computed from that list.  The list of
applicable methods may have been modified in that an accessor method
may have been altered in order to account for the location of the slot
being accessed, as a function of the information in the entry key.
This information is bundled up onto a CONS cell, where the list of
applicable methods is in the CAR and the effective method function is
in the CDR. Several entries may share this information.  When that is
the case, their CONS cells are identical (EQ).

The additional information of an entry also includes a MASK.  The mask
is similar to the key, but it contains the symbol * in a position that
is not specialized upon by any of the methods in the list of
applicable methods of the entry.  The mask is what is used to compute
the discriminating function.

Entries in the call history may be removed as a result of changes in
the class hierarchy, or as a result of a method being added or removed
from the generic function.  How the call history is updated in these
cases is described elsewhere.

Entries in the call history are added as a result of a dispatch miss
that involves arguments that correspond to applicable methods, but
that have not yet been encountered.  When a dispatch-miss happens, we
first call COMPUTE-APPLICABLE-METHODS-USING-CLASSES, passing it the
list of classes of the arguments.  This call may result in a second
return value (called OK) that is either true or false.
