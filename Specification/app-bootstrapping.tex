\chapter{Bootstrapping principles}

In this appendix, we describe general principles of bootstrapping, as
opposed to implementation details.

\section{General restrictions}

We define the \emph{purity} of some object to be a non-negative
integer.  An object of purity $p$ is an instance of a class of purity
$p-1$.  An object of purity $0$ is a \emph{host object}.  An object of
purity $1$ is a is a \emph{bridge object}.  An object of purity $2$ is
an \emph{impure ersatz object}.  An object of purity $3$ or more is a
\emph{pure ersatz object}.  Each \emph{phase} of the bootstrapping
procedure creates objects of one purity.

Suppose we want to access some part of a generic function metaobject
of purity $p$.  Perhaps we want to add methods to it, or set its
discriminating function.  During bootstrapping, such access must be
done by fully functioning generic functions.  For that reason, we use
functions of purity $p-1$ for such access, and we can assume that when
we need to accomplish such access to a generic function of purity $p$,
then either the accessors of purity $p-1$ are either already fully
functional, or the machinery for making them fully functional is fully
functional, so that we can freely use generic functions of purity
$p-1$ to access a generic function of purity $p$.

More generally, we assume that the arguments to a generic function of
purity $p$ are objects of purity $p+1$.  It follows that the
\emph{specializers} of a generic function of purity $p$ are objects of
purity $p$ as well, since they are typically the classes of the
arguments.  Since the arguments have purity $p+1$, their classes are
objects of purity $p$.  Furthermore we want the methods of a generic
function of purity $p$ to be objects of purity $p$, and we want the
method functions of those methods to be objects of purity $p$.  Idem
for the \emph{method combination}, the \emph{effective method
  functions}, and the \emph{discriminating function}.  It follows that
the \emph{generic-function class} and the \emph{method class} of a
generic function of purity $p$ are objects of purity $p-1$.

In general, we would like for a function of purity $p$ to call other
functions of purity $p$, but during bootstrapping we can not always
accomplish this restriction.  As a general principle, however, we want
to minimize the exceptions to this rule, because these exceptions
require specific bootstrapping code to be handled correctly.

For classes of purity $p$ we want the the direct superclasses and its
direct and effective slots to be objects of purity $p$ as well.

\section{Object allocation}
\label{sec-app-bootstrapping-object-allocation}

Suppose we want to allocate an object of purity $p$.  To do so, we
need to instantiate a class of purity $p-1$.  Instantiating a class
involves calling \texttt{allocate-instance}, which is a generic
function.  Since \texttt{allocate-instance} is a generic function that
takes an object of purity $p-1$ as an argument, it must be a generic
function of purity $p-2$.  Furthermore, \texttt{allocate-instance}
must call accessor generic functions to inspect the class metaobject
in order to determine the \emph{unique number} of the class as well as
its list of slot metaobjects, and information about the size of its
instances.  All these accessors are objects of purity $p-2$ as well.

Now \texttt{allocate-instance} was called by \texttt{make-instance}
which is a generic function.  Assume for the moment that
\texttt{make-instance} was given a class metaobject to instantiate.
Then \texttt{make-instance} must also be of purity $p-2$.

Another interesting aspect of \texttt{make-instance} is that, if it is
given a symbol as opposed to a class metaobject, it must call
\texttt{find-class}.  Now \texttt{find-class} is probably an ordinary
function, and it must find a class metaobject of purity $p-1$.  If we
assume that \texttt{find-class} has the same purity as
\texttt{make-instance}, then we have a function of purity $p-2$ that
must find a class metaobject of purity $p-1$.  This information may
determine in which environment we decide to allocate functions and
classes of different purity.

\begin{figure}
\begin{center}
\inputfig{fig-make-instance.pdf_t}
\end{center}
\caption{\label{fig-make-instance}
Object allocation.}
\end{figure}

\section{Object initialization}

After \texttt{make-instance} has called \texttt{allocate-instance}, it
calls \texttt{initialize-instance}, giving it the newly allocated
instance and some key/value pairs.  \texttt{initialize-instance} is
thus a generic function that takes an object of purity $p$, and it
follows that \texttt{initialize-instance} is a generic function of
purity $p-1$.  We therefore have a case where a generic function
(i.e. \texttt{make-instance}) of purity $p-2$ calls a generic function
of purity $p-1$.

\section{Processing the \texttt{defclass} macro}

The expansion of the \texttt{defclass} macro results in a call to
\texttt{ensure-class} which is an ordinary function that immediately
calls the generic function \texttt{ensure-class-using-class}.  The
\textit{class} argument to \texttt{ensure-class-using-class} is either
\texttt{nil} if the class does not exist, or the class metaobject to
be reinitialized.  Thus, if the new or the existing class is an object
of purity $p$, then \texttt{ensure-class} and
\texttt{ensure-class-using-class} should be of purity $p-1$.
The AMOP states that the \emph{direct-superclasses} argument to
\texttt{defclass} becomes the value of the
\texttt{:direct-superclasses} argument to \texttt{ensure-class} so
there is no call to \texttt{find-class} involved here.

The most interesting aspect of the \texttt{defclass} macro is the
conversion of a slot specification to a \emph{canonicalized slot
  specification}, and specifically the conversion of the
\texttt{:initform} option to the value of the \texttt{:initfunction}
option.  This conversion is done by \texttt{compile} that takes the
initform wrapped in a \texttt{lambda} expression and turns it into a
function.  So \texttt{compile} in this case, must build a function of
purity $p$ since it is going to become part of a slot-definition
metaobject of that purity.

The \texttt{:metaclass} option to the \texttt{defclass} macro becomes
the value of the \texttt{:metaclass} keyword argument to
\texttt{ensure-class}, so no conversion is involved.

The keyword argument \texttt{:direct-superclasses} passed to
\texttt{ensure-class-using-class} may contain class names or class
metaobjects.  If it contains a class name, it is converted to a class
metaobject.  It must do so by calling \texttt{find-class} or something
similar.  So preferably, \texttt{find-class} has purity $p-1$ as well,
and it must find a class with purity $p$.  This behavior is consistent
with the \texttt{find-class} we encountered in
\refSec{sec-app-bootstrapping-object-allocation}.

The keyword argument \texttt{:metaclass} passed to
\texttt{ensure-class-using-class} may contain a class name or a class
metaobject.  If it contains a class name, it is converted to a class
metaobject.  It must do so by calling \texttt{find-class} or something
similar.  In this case, \texttt{find-class} must find a class with
purity $p-1$ which is in direct conflict with what it must do for the
\texttt{:direct-superclasses} option.  We solve this problem in
\sysname{} by using an indirection called \texttt{find-metaclass} that
does what is needed.

\section{Initialization of class metaobjects}

In \sysname{}, class initialization is accomplished by an
\texttt{:around} method on \texttt{shared-initialize}.  It calls an
ordinary function to accomplish its task.  If the class metaobject to
be initialized has purity $p$, then \texttt{shared-initialize} has
purity $p-1$.

The \texttt{:direct-slots} argument is a list of canonicalized slot
specifications.  Each element is converted to a direct slot definition
metaobject.  This conversion is done in two steps.  First the generic
function \texttt{direct-slot-definition-class} is called, passing the
class metaobject as an argument.  Therefore
\texttt{direct-slot-definition-class} is a generic function of purity
$p-1$, just like \texttt{shared-initialize}.  Second,
\texttt{make-instance} is called with the class returned by
\texttt{direct-slot-definition-class}, and the canonicalized
slot specification.  So, here \texttt{make-instance} is called on a
class of purity $p$.  Therefore \texttt{make-instance} is of purity
$p-1$.
