\chapter{List of tasks of limited size}

In this chapter, we give a list of tasks that can be accomplished in a
shorter period of time, typically between a few days and a few weeks.
The tasks in this list are meant for people who would like to
contribute to \sysname{}, but who either lack a significant amount of
time, or the knowledge, to intervene in more complex tasks.

Each task in this list is meant to be interesting to the person who
decides to take it on.  Thus, we have avoided trivial tasks such as
untabifying source code, fixing grammar in comments, and generally
altering code to conform to the guidelines in the previous two
chapters.

\section{Implement hash tables}

Hash tables have not yet been implemented in \sysname{}.  We would
like to investigate several possible implementations, and perhaps
propose several such implementations, with different characteristics,
in the code base of \sysname{}.

Since \sysname{} has a very efficient technique for generic dispatch,
we think that there could be a hierarchy of classes with different
characteristics, all implementing the standard protocol for hash
tables specified by the \commonlisp{} standard.

In general, we want hash tables to be thread safe.  If it is possible
to make them lock free, that is even better.

The code should be structured so that it looks ``natural'' in
intrinsic setting, i.e. when the code is loaded into \sysname{}.
However, we would like for the code to be structured such that it can
be tested in an extrinsic setting as well.

The code should contain a separate \texttt{test} system, probably using
extensive testing through the use of randomly generated operations.

\subsection{Implementation using open hashing}

\subsection{Implementation using vector buckets}

\section{Implement streams}

Streams have non yet been implemented in \sysname{}.  The full set of
streams defined by the standard must be implemented at some point.

Initially, we are not concerned with extreme performance
requirements.  For that reason, we think that the protocol functions
can be implemented on top of the Gray-streams protocol.

The code should be structured so that it looks ``natural'' in
intrinsic setting, i.e. when the code is loaded into \sysname{}.
However, we would like for the code to be structured such that it can
be tested in an extrinsic setting as well.

\section{Better error messages for the \texttt{loop} module}

The parser for the \texttt{loop} macro uses a home-grown version of
combinator parsing.  Since \texttt{loop} clauses do not need
backtracking, this feature of normal combinator parsing is not
included.

However, we think that combinator parsing is perhaps not a well suited
technique when good error messages are a requirement.

This task consists of investigating whether good error messages can be
obtained with the current technique, while maintaining a reasonable
structure of the code.  Or whether some other technique might be
better suited.  We are thinking that Earley parsong might be a good
candidate.

\section{Better error messages by the lambda-list parser}

The parser for the \texttt{loop} macro uses a home-grown
implementation of the Earley parsing technique.  In theory, this
technique should be excellent when it comes to generating good error
messages in case of parse failures.  However, we have not implemented
such error messages yet.

This task consists of adding such error messages.
