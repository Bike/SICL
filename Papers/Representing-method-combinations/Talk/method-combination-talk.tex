\documentclass{beamer}
\usepackage[latin1]{inputenc}
\beamertemplateshadingbackground{red!10}{blue!10}
%\usepackage{fancybox}
\usepackage{epsfig}
\usepackage{verbatim}
\usepackage{url}
%\usepackage{graphics}
%\usepackage{xcolor}
\usepackage{fancybox}
\usepackage{moreverb}
%\usepackage[all]{xy}
\usepackage{listings}
\usepackage{filecontents}
\usepackage{graphicx}

\lstset{
  language=Lisp,
  basicstyle=\scriptsize\ttfamily,
  keywordstyle={},
  commentstyle={},
  stringstyle={}}

\def\inputfig#1{\input #1}
\def\inputeps#1{\includegraphics{#1}}
\def\inputtex#1{\input #1}

\inputtex{logos.tex}

%\definecolor{ORANGE}{named}{Orange}

\definecolor{GREEN}{rgb}{0,0.8,0}
\definecolor{YELLOW}{rgb}{1,1,0}
\definecolor{ORANGE}{rgb}{1,0.647,0}
\definecolor{PURPLE}{rgb}{0.627,0.126,0.941}
\definecolor{PURPLE}{named}{purple}
\definecolor{PINK}{rgb}{1,0.412,0.706}
\definecolor{WHEAT}{rgb}{1,0.8,0.6}
\definecolor{BLUE}{rgb}{0,0,1}
\definecolor{GRAY}{named}{gray}
\definecolor{CYAN}{named}{cyan}

\newcommand{\orchid}[1]{\textcolor{Orchid}{#1}}
\newcommand{\defun}[1]{\orchid{#1}}

\newcommand{\BROWN}[1]{\textcolor{BROWN}{#1}}
\newcommand{\RED}[1]{\textcolor{red}{#1}}
\newcommand{\YELLOW}[1]{\textcolor{YELLOW}{#1}}
\newcommand{\PINK}[1]{\textcolor{PINK}{#1}}
\newcommand{\WHEAT}[1]{\textcolor{wheat}{#1}}
\newcommand{\GREEN}[1]{\textcolor{GREEN}{#1}}
\newcommand{\PURPLE}[1]{\textcolor{PURPLE}{#1}}
\newcommand{\BLACK}[1]{\textcolor{black}{#1}}
\newcommand{\WHITE}[1]{\textcolor{WHITE}{#1}}
\newcommand{\MAGENTA}[1]{\textcolor{MAGENTA}{#1}}
\newcommand{\ORANGE}[1]{\textcolor{ORANGE}{#1}}
\newcommand{\BLUE}[1]{\textcolor{BLUE}{#1}}
\newcommand{\GRAY}[1]{\textcolor{gray}{#1}}
\newcommand{\CYAN}[1]{\textcolor{cyan }{#1}}

\newcommand{\reference}[2]{\textcolor{PINK}{[#1~#2]}}
%\newcommand{\vect}[1]{\stackrel{\rightarrow}{#1}}

% Use some nice templates
\beamertemplatetransparentcovereddynamic

\newcommand{\A}{{\mathbb A}}
\newcommand{\degr}{\mathrm{deg}}

\title{Representing method combinations}

\author{Robert Strandh}
\institute{
LaBRI, University of Bordeaux
}
\date{April, 2020}

%\inputtex{macros.tex}


\begin{document}
\frame{
\resizebox{3cm}{!}{\includegraphics{Logobx.pdf}}
\hfill
\resizebox{1.5cm}{!}{\includegraphics{labri-logo.pdf}}
\titlepage
\vfill
\small{European Lisp Symposium, Z�rich, Switzerland \hfill ELS2020}
}

\setbeamertemplate{footline}{
\vspace{-1em}
\hspace*{1ex}{~} \GRAY{\insertframenumber/\inserttotalframenumber}
}

\frame{
\frametitle{Context: The \sicl{} project}

https://github.com/robert-strandh/SICL

Several objectives:

\begin{itemize}
\item Create high-quality \emph{modules} for implementors of
  \commonlisp{} systems.
\item Improve existing techniques with respect to algorithms and data
  structures where possible.
\item Improve readability and maintainability of code.
\item Improve documentation.
\item Ultimately, create a new implementation based on these modules.
\end{itemize}
}

\frame{
\frametitle{Generic function}
A feature of object-oriented programming languages.
\vskip 0.25cm
Can be thought of as a collection of \emph{methods} sharing the same
name.
\vskip 0.25cm
In traditional languages, a generic function is not a first-class object.
\vskip 0.25cm
In \commonlisp{} a generic function is an instance of a subclass of
\texttt{function}, so it is a first-class object.
}

\frame{
\frametitle{Generic-function invocation}
In traditional languages: \texttt{<arg1>.gf(<arg2>, ...)}
\vskip 0.25cm
In \commonlisp{} \texttt{(gf <arg1> <arg2> ...)}
\vskip 0.25cm
Because of subclassing, an invocation may result in several methods
being \emph{applicable}
}

\frame{
\frametitle{\texttt{compute-effective-method}}
This generic function has three parameters:
\begin{enumerate}
\item A generic-function metaobject
\item A method-combination metaobject
\item A list of method metaobjects
\end{enumerate}
\vskip 0.25cm
A call to this generic function returns an an effective method.
}


\frame{
\frametitle{Role of the method-combination metaobject}
In \texttt{compute-effective-method}, the method-combination
metaobject designates a \emph{method-combination procedure}.
\vskip 0.25cm
This procedure can be implemented in different ways:
\begin{itemize}
\item As a method on \texttt{compute-effective-method}, specialized to
  some method-combination class.
\item As a function stored in the method-combination object.
\item Any other way that will accomplish the task.
\end{itemize}
}

\frame{
\frametitle{Title}
Stuff
\vskip 0.25cm
More stuff
}

\frame[containsverbatim]{
\frametitle{A protocol for first-class global environments}
\begin{verbatim}
(cl:in-package #:sicl-genv)

(defgeneric fboundp (function-name environment))

(defgeneric fmakunbound (function-name environment))
\end{verbatim}

}

\frame{
\frametitle{Future work}

\begin{itemize}
\item stuff
\item stuff
\item stuff
\end{itemize}

}

\frame{
  \frametitle{Acknowledgments}

We would like to thank ...
}

\frame{
\frametitle{Thank you}
}

%% \frame{\tableofcontents}
%% \bibliography{references}
%% \bibliographystyle{alpha}

\end{document}
