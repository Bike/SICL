\documentclass{beamer}
\usepackage[utf8]{inputenc}
\beamertemplateshadingbackground{red!10}{blue!10}
%\usepackage{fancybox}
\usepackage{epsfig}
\usepackage{verbatim}
\usepackage{url}
%\usepackage{graphics}
%\usepackage{xcolor}
\usepackage{fancybox}
\usepackage{moreverb}
%\usepackage[all]{xy}
\usepackage{listings}
\usepackage{filecontents}
\usepackage{graphicx}

\lstset{
  language=Lisp,
  basicstyle=\scriptsize\ttfamily,
  keywordstyle={},
  commentstyle={},
  stringstyle={}}

\def\inputfig#1{\input #1}
\def\inputeps#1{\includegraphics{#1}}
\def\inputtex#1{\input #1}

\inputtex{logos.tex}

%\definecolor{ORANGE}{named}{Orange}

\definecolor{GREEN}{rgb}{0,0.8,0}
\definecolor{YELLOW}{rgb}{1,1,0}
\definecolor{ORANGE}{rgb}{1,0.647,0}
\definecolor{PURPLE}{rgb}{0.627,0.126,0.941}
\definecolor{PURPLE}{named}{purple}
\definecolor{PINK}{rgb}{1,0.412,0.706}
\definecolor{WHEAT}{rgb}{1,0.8,0.6}
\definecolor{BLUE}{rgb}{0,0,1}
\definecolor{GRAY}{named}{gray}
\definecolor{CYAN}{named}{cyan}

\newcommand{\orchid}[1]{\textcolor{Orchid}{#1}}
\newcommand{\defun}[1]{\orchid{#1}}

\newcommand{\BROWN}[1]{\textcolor{BROWN}{#1}}
\newcommand{\RED}[1]{\textcolor{red}{#1}}
\newcommand{\YELLOW}[1]{\textcolor{YELLOW}{#1}}
\newcommand{\PINK}[1]{\textcolor{PINK}{#1}}
\newcommand{\WHEAT}[1]{\textcolor{wheat}{#1}}
\newcommand{\GREEN}[1]{\textcolor{GREEN}{#1}}
\newcommand{\PURPLE}[1]{\textcolor{PURPLE}{#1}}
\newcommand{\BLACK}[1]{\textcolor{black}{#1}}
\newcommand{\WHITE}[1]{\textcolor{WHITE}{#1}}
\newcommand{\MAGENTA}[1]{\textcolor{MAGENTA}{#1}}
\newcommand{\ORANGE}[1]{\textcolor{ORANGE}{#1}}
\newcommand{\BLUE}[1]{\textcolor{BLUE}{#1}}
\newcommand{\GRAY}[1]{\textcolor{gray}{#1}}
\newcommand{\CYAN}[1]{\textcolor{cyan }{#1}}

\newcommand{\reference}[2]{\textcolor{PINK}{[#1~#2]}}
%\newcommand{\vect}[1]{\stackrel{\rightarrow}{#1}}

% Use some nice templates
\beamertemplatetransparentcovereddynamic

\newcommand{\A}{{\mathbb A}}
\newcommand{\degr}{\mathrm{deg}}

\title{Creating a \commonlisp{} implementation\\(Part 2)}

\author{Robert Strandh}
\institute{
}
\date{June, 2020}

%\inputtex{macros.tex}

\begin{document}
\frame{
\titlepage
}

\setbeamertemplate{footline}{
\vspace{-1em}
\hspace*{1ex}{~} \GRAY{\insertframenumber/\inserttotalframenumber}
}

\frame{
\frametitle{Recapitualation, strategy 1}
\vskip 0.25cm
\begin{itemize}
\item A \emph{core} written in an existing language, typically C.
\item Additional modules written in \commonlisp{} added to the core.
\end{itemize}
}

\frame{
\frametitle{Recapitualation, strategy 1}
\vskip 0.25cm
The core may contain more than what we would have liked:
\vskip 0.25cm
\begin{itemize}
\item An \emph{inner core}:
  \begin{itemize}
  \item Object allocation.
  \item Memory management.
  \item Accessors for built-in object types.
  \end{itemize}
\item An \emph{extended core}:
  \begin{itemize}
  \item Functions placed here in order to break circular
    dependencies.
  \item Temporary or permanent functions required for bootstrapping,
    like \texttt{read} and \texttt{eval}.
  \end{itemize}
\end{itemize}
}

\frame{
\frametitle{Recapitulation, strategy 1}
\vskip 0.25cm
\begin{figure}
\begin{center}
\inputfig{fig-recapitulation-strategy-1.pdf_t}
\end{center}
\end{figure}
}

\frame[containsverbatim]{
\frametitle{Recapitulation, strategy 1}
\vskip 0.25cm
The inner core may contain functions at a level lower than the lowest
level of \commonlisp{}.  Example:
\begin{verbatim}
(defun car (object)
  (if (consp object)
      (core:cons-car object)
      (if (null object)
          nil
          (error ...))))
\end{verbatim}
\vskip 0.25cm
Here, \texttt{car} is part of the \commonlisp{} code, and
\texttt{cons-car} is part of the inner core.
}

\frame{
\frametitle{Dependencies, execution}
\vskip 0.25cm
This relation means that the \emph{execution} of A may
invoke B.
\begin{figure}
\begin{center}
\inputfig{fig-dependencies-1.pdf_t}
\end{center}
\end{figure}
\vskip 0.25cm
Executing a macro means executing the macro function, which is usually
done by the compiler at compile time.
}

\frame[containsverbatim]{
\frametitle{Example}
\vskip 0.25cm
{\small\begin{verbatim}
(defmacro case (expression &rest clauses)
  (let ((expression-var (gensym)))
    `(let ((,expression-var ,expression))
       (cond ,@(mapcar (lambda (clause)
                         `((eql ,expression-var
                                ',(first clause))
                           ,@(rest clause)))
                       clauses)))))
\end{verbatim}
\begin{figure}
\begin{center}
\inputfig{fig-dependencies-3.pdf_t}
\end{center}
\end{figure}
}
}

\frame{
\frametitle{Dependencies, compilation}
\vskip 0.25cm
This relation means that the \emph{compilation} of A may
invoke B.
\begin{figure}
\begin{center}
\inputfig{fig-dependencies-2.pdf_t}
\end{center}
\end{figure}
}

\frame{
\frametitle{Abstract machine}
\vskip 0.25cm
\begin{figure}
\begin{center}
\inputfig{fig-abstract-machine-1.pdf_t}
\end{center}
\end{figure}
}


\end{document}
