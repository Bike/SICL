\documentclass{beamer}
\usepackage[utf8]{inputenc}
\beamertemplateshadingbackground{red!10}{blue!10}
%\usepackage{fancybox}
\usepackage{epsfig}
\usepackage{verbatim}
\usepackage{url}
%\usepackage{graphics}
%\usepackage{xcolor}
\usepackage{fancybox}
\usepackage{moreverb}
%\usepackage[all]{xy}
\usepackage{listings}
\usepackage{filecontents}
\usepackage{graphicx}

\lstset{
  language=Lisp,
  basicstyle=\scriptsize\ttfamily,
  keywordstyle={},
  commentstyle={},
  stringstyle={}}

\def\inputfig#1{\input #1}
\def\inputeps#1{\includegraphics{#1}}
\def\inputtex#1{\input #1}

\inputtex{logos.tex}

%\definecolor{ORANGE}{named}{Orange}

\definecolor{GREEN}{rgb}{0,0.8,0}
\definecolor{YELLOW}{rgb}{1,1,0}
\definecolor{ORANGE}{rgb}{1,0.647,0}
\definecolor{PURPLE}{rgb}{0.627,0.126,0.941}
\definecolor{PURPLE}{named}{purple}
\definecolor{PINK}{rgb}{1,0.412,0.706}
\definecolor{WHEAT}{rgb}{1,0.8,0.6}
\definecolor{BLUE}{rgb}{0,0,1}
\definecolor{GRAY}{named}{gray}
\definecolor{CYAN}{named}{cyan}

\newcommand{\orchid}[1]{\textcolor{Orchid}{#1}}
\newcommand{\defun}[1]{\orchid{#1}}

\newcommand{\BROWN}[1]{\textcolor{BROWN}{#1}}
\newcommand{\RED}[1]{\textcolor{red}{#1}}
\newcommand{\YELLOW}[1]{\textcolor{YELLOW}{#1}}
\newcommand{\PINK}[1]{\textcolor{PINK}{#1}}
\newcommand{\WHEAT}[1]{\textcolor{wheat}{#1}}
\newcommand{\GREEN}[1]{\textcolor{GREEN}{#1}}
\newcommand{\PURPLE}[1]{\textcolor{PURPLE}{#1}}
\newcommand{\BLACK}[1]{\textcolor{black}{#1}}
\newcommand{\WHITE}[1]{\textcolor{WHITE}{#1}}
\newcommand{\MAGENTA}[1]{\textcolor{MAGENTA}{#1}}
\newcommand{\ORANGE}[1]{\textcolor{ORANGE}{#1}}
\newcommand{\BLUE}[1]{\textcolor{BLUE}{#1}}
\newcommand{\GRAY}[1]{\textcolor{gray}{#1}}
\newcommand{\CYAN}[1]{\textcolor{cyan }{#1}}

\newcommand{\reference}[2]{\textcolor{PINK}{[#1~#2]}}
%\newcommand{\vect}[1]{\stackrel{\rightarrow}{#1}}

% Use some nice templates
\beamertemplatetransparentcovereddynamic

\newcommand{\A}{{\mathbb A}}
\newcommand{\degr}{\mathrm{deg}}

\title{Creating a Common Lisp implementation\\(Part 1)}

\author{Robert Strandh}
\institute{
}
\date{June, 2020}

%\inputtex{macros.tex}

\begin{document}
\frame{
\titlepage
}

\setbeamertemplate{footline}{
\vspace{-1em}
\hspace*{1ex}{~} \GRAY{\insertframenumber/\inserttotalframenumber}
}

\frame{
\frametitle{Compiler for a traditional batch language}
\vskip 0.25cm
\begin{figure}
\begin{center}
\inputfig{fig-c-unix-1.pdf_t}
\end{center}
\end{figure}
}

\frame{
\frametitle{Compiler for a traditional batch language}
\vskip 0.25cm
Characteristics:
\vskip 0.25cm
\begin{itemize}
\item Macros and declarations (implicit or explicit) are entered into
  the environment.
\item The compiler uses the environment to emit warnings, and to
  determine how to generate code.
\end{itemize}
}

\frame{
\frametitle{Run-time support for a traditional batch language}
\vskip 0.25cm
\begin{figure}
\begin{center}
\inputfig{fig-c-unix-2.pdf_t}
\end{center}
\end{figure}
}

\frame{
\frametitle{Run-time support for a traditional batch language}
\vskip 0.25cm
Characteristics:
\vskip 0.25cm
\begin{itemize}
\item Each program executes in a separate address space
\item Systems calls are used for file I/O, communication between
  programs, configuration, etc.
\item Communication between programs uses pipes, requiring transitions
  through the kernel.
\end{itemize}
}

\frame{
\frametitle{Common Lisp file compiler}
\vskip 0.25cm
\begin{figure}
\begin{center}
\inputfig{fig-common-lisp-file-compilation.pdf_t}
\end{center}
\end{figure}
}

\frame{
\frametitle{Common Lisp file compiler}
\vskip 0.25cm
Characteristics:
\vskip 0.25cm
\begin{itemize}
\item Requires and existing Common Lisp implementation (at least partial).
\item more?
\end{itemize}
}

\frame{
\frametitle{Title}
\vskip 0.25cm
Stuff
\vskip 0.25cm
Stuff
}

\frame{
\frametitle{Title}
%% \vskip 0.25cm
%% \begin{figure}
%% \begin{center}
%% \inputfig{fig-50.pdf_t}
%% \end{center}
%% \end{figure}
}

\frame{
\frametitle{Thank you}
}

\end{document}
